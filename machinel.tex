La definición de aprendizaje maquinal de Tom Mitchell dice que ``un programa de computadora aprende de experiencia $E$ con respecto a una tarea $T$ y una medición de rendimiento $P$, si su rendimiento en $T$, medido por $P$, mejora con experiencia $E$.''

Esto dignifica que a diferencia del paradigma clásico de programación, donde los humanos introducen órdenes y datos para ser procesados de acuerdo con dichas reglas, en el aprendizaje maquinal el humano introduce datos y respuestas esperadas de estos datos ``y el producto son las reglas''*.\notad{``and out come the rules''}.

Si no es ser programado explícitamente, entonces un sistema de aprendizaje maquinal es entrenado: se le presentan muchos ejemplos relevantes a una tarea, y si encuentra una estructura estadística en ellos, genera reglas para automatizar la tarea.