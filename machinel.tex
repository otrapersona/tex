A diferencia del paradigma clásico de programación, donde los humanos introducen órdenes y datos para ser procesados de acuerdo con dichas reglas, en el aprendizaje maquinal el humano introduce datos y respuestas esperadas de estos datos ``y el producto son las reglas''*.\notad{``and out come the rules''}.

Tom Mitchel 97:
Se dice que un programa de computadora aprende de experiencia $E$ con respecto a una tarea $T$ y una medición de desempeño $P$, si su desempeño $T$, medido por $P$, mejora con experiencia $E$.

En vez de ser programado explícitamente, un sistema de aprendizaje maquinal es entrenado: se le presentan muchos ejemplos relevantes a una tarea, y si encuentra una estructura estadística en ellos, genera reglas para automatizar la tarea.