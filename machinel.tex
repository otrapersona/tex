A diferencia del paradigma clásico de programación, donde los humanos introducen órdenes y datos para ser procesados de acuerdo con dichas reglas, en el aprendizaje maquinal el humano introduce datos y respuestas esperadas de estos datos ``y el producto son las reglas''* \notad{``and out come the rules''}. En vez de ser programado explícitamente, un sistema de aprendizaje maquinal es entrenado: se le presentan muchos ejemplos relevantes a una tarea, y si encuentra una estructura estadística en ellos, le permitirán*\notadhi{tiempos?} generar reglas
\\
\\
dl with françois p5