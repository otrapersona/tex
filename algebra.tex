Pegar texto de gdocs\\
Operaciones unarias: $\mu(R) \rightarrow R^\prime$\\
Operaciones binarias: $\beta(P,Q) \rightarrow R$\\
Relaciones: $P,Q,R,R^\prime$
--


Los cimientos del modelo relacional son el \emph{álgebra relacional}, las operaciones del álgebra producen nuevas relaciones, que pueden manipularse también por medio de operaciones del álgebra mismo. Una secuencia de operaciones de álgebra relacional forma una \emph{expresión de álgebra relacional} cuyo resultado es una relación que representa el resultado de una consulta (o solicitud de consulta) de base de datos. Estas operaciones pueden clasificar en dos grupos, operaciones de conjuntos*\notad{``set operations''} de la teoría de conjuntos matemáticos refiere a la operaciones \emph{UNION}, \emph{INTERSECTION}, \emph{SET DIFFERENCE} y \emph{CARTESIAN PRODUCT} (también llamado \emph{CROSS PRODUCT}). El otro grupo consiste en operaciones específicas para pases de batos relacionales: \emph{SELECT}, \emph{PROJECT} y \emph{JOIN}.

