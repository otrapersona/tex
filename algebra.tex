Los cimientos del modelo relacional son el \emph{álgebra relacional}, las operaciones del álgebra producen nuevas relaciones, que pueden manipularse también por medio de operaciones del álgebra mismo. Una secuencia de operaciones de álgebra relacional forma una \emph{expresión de álgebra relacional} cuyo resultado es una relación que representa el resultado de una consulta (o solicitud de consulta) de base de datos. Estas operaciones pueden clasificar en dos grupos, operaciones de conjuntos*\notad{(así se llaman? checar libro en l'espagnol)``set operations''} de la teoría de conjuntos matemáticos refiere a la operaciones \emph{UNION}, \emph{INTERSECTION}, \emph{SET DIFFERENCE} y \emph{CARTESIAN PRODUCT} (también llamado \emph{CROSS PRODUCT}). El otro grupo consiste en operaciones específicas para bases de datos relacionales: \emph{JOIN}, \emph{SELECT} y \emph{PROJECT}. Estas dos últimas, por operar con una sola relación, son también conocidas como \emph{operaciones unarias}.

SELECT es utilizada para elegir un subconjunto de tuplas de una relación que satisfacen sus condiciones
\begin{equation}
\sigma_\text{\textless condición de selección\textgreater}{\text{(R)}}
\end{equation}
donde $\sigma$ denota el operador SELECT, y la condición de selección es una expresión booleana especificada en los atributos de la relación $R$.

Si pensamos en las relaciones como una tabla, SELECT es utilizado para elegir filas horizontalmente. En contraste, la operación unaria PROJECT elige atributos verticalmente (columnas, en el caso una tabla) que satisfacen sus condiciones
\begin{equation}
\pi_\text{\textless lista de atributos\textgreater}{\text{(R)}}
\end{equation}
donde $\pi$ es el denota el operador PROJECT y la lista de atributoses la sublista deseada de atributos de los atributos de la relación $R$.
