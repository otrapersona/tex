\renewcommand{\abstractname}{Resumen ejecutivo.}\begin{abstract}
El uso de las redes sociales en el mundo ha generado la aparición de estudios enfocados a un sinnúmero de casos y aplicaciones.

Estos estudios van siempre acompañados con técnicas de \emph{procesamiento de lenguaje natural}, unos más sofisticados y otros más básicos; a partir de palabras simples, considerando cierto tipo de personajes, el uso de ``hashtag", diferentes rangos de fechas, con enfoque de sentimientos.

Pero de qué sirve el procesamiento de lenguaje natural sin un modelo estadístico que describa numéricamente lo que está sucediendo con las muestras de textos, que indique qué hay en común en la opinión de la gente o en los reporteros, o cuál es la tendencia social.

La estadística en esta propuesta de modelo tiene dos intervenciones, la primera es al inicio del proceso, con una estimación de lo que se pretende obtener, la segunda es con una comprobación de dicha estimación. En su caso una retroalimentación para asegurar ciertos valores usando aprendizaje maquinal y ajustando parámetros de categorización y diversificación de palabras, buscando confirmar o rechazar una suposición, con elementos numéricos bien fundamentados.
\end{abstract}
