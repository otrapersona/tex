\section {Objetivos}
\subsection {General}
Haciendo uso de las ciencias de la computación, las herramientas matemáticas de estadística y métodos de aprendizaje autónomo, se busca obtener información cuantitativa de textos provenientes de redes sociales, cadenas noticiosas y audio de programas de capacitación, para inferir posiciones, tendencias, comportamientos o razones de grupos sociales, considerando un ciclo de clasificación, estimación, detección y comprobación.
\subsection {Particulares}
\begin{enumerate}
    \item Desarrollar un modelo de base de datos que permita la captura de categorías para un determinado problema, los elementos de identificación de cada categoría, el origen de la información y su correlación.
    \item Construir una estructura de datos que capte la estimación o valores esperados para el procesamiento de textos.
    \item Elaborar un sistema de objetos para el soporte de los elementos de aprendizaje autónomo.
    \item Generar los elementos de captura de textos para su almacenamiento y procesamiento.
    \item Elaborar un modelo estadístico que permita comprobar las estimaciones a partir de los datos y en consecuencia realizar un ajuste en los parámetros usados para el aprendizaje autónomo.
    \item Producir los reportes con un análisis estadístico que faciliten la interpretación de resultados y den pauta para la obtención del conocimiento de interés.
\end{enumerate}
\subsection {Metas científicas}
Metas científicas