\section {Objetivos}
\subsection {General}
Haciendo uso de las ciencias de la computación, las herramientas matemáticas de estadística y métodos de aprendizaje autónomo, se busca obtener información cuantitativa de textos. Las fuentes de información será redes sociales, cadenas noticiosas y audios de programas de capacitación; para inferir posiciones, tendencias, comportamientos o razones de grupos sociales. El ciclo trabajo es de clasificación, estimación, detección y comprobación.
\subsection {Particulares}
\begin{enumerate}
    \item Desarrollar un modelo de base de datos que permita la captura de categorías para un determinado problema, los elementos de identificación de cada categoría, el origen de la información y su correlación.
    \item Construir una estructura de datos que capte la estimación o valores esperados para el procesamiento de textos.
    \item Elaborar un sistema de objetos para el soporte de los elementos de aprendizaje autónomo.
    \item Generar los elementos de captura de textos para su almacenamiento y procesamiento.
    \item Elaborar un modelo estadístico que permita comprobar las estimaciones a partir de los datos y en consecuencia realizar un ajuste en los parámetros usados para el aprendizaje autónomo.
    \item Producir los reportes con un análisis estadístico que faciliten la interpretación de resultados y den pauta para la obtención del conocimiento de interés.
\end{enumerate}
\section {Metas científicas}
La meta de este proyecto es la integración de elementos de estadística, ciencias computacionales, aprendizaje autónomo (machine learning) para el procesamiento de lenguaje natural (PLN). Aprovechando el potencial computacional se analizará el alcance de los modelos estadísticos para la evaluación comparativa de los parámetros; este proceso se puede hacer en forma manual, sin embargo, por el volumen de datos se hará uso de las tecnologías computacionales.
Esta metodología usualmente se trabaja de forma aislada, en este proyecto se busca integrar el procedimiento con fundamentación estadística. En un principio se considerará el modelo de regresión lineal, posteriormente se analizarán la implementación de otros modelos de regresión no lineales.