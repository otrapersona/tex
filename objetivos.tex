\section {Objetivos}\label{sec:objetivos}
\subsection {General}
Haciendo uso de las ciencias de la computación, las herramientas matemáticas de estadística y métodos de aprendizaje autónomo, se busca obtener información cuantitativa de textos provenientes de redes sociales, cadenas noticiosas y audio de programas de capacitación, para inferir posiciones, tendencias, comportamientos o razones de grupos sociales, considerando un ciclo de clasificación, estimación, detección y comprobación.
\subsection {Particulares}
Revisé algunas tesis para darme la idea, necesito saber un poco más del tema para desarrollar esta parte, creo que sería algo así:
\begin{enumerate}
    \item Investigación
    \item Análisis info
    \item Diseño/desarrollo
    \item Comprobación
\end{enumerate}

\#\#\#\#\#\#\#\#\#\#\#\#\#\#\#\#\#\#\#\#\#\#\#\#\#\#\#\#\#\#\#\#\#\#\#\#\#\#\#\#\#\\

Al integrar\\
/tecnologías/técnicas/modelos/librerías/frameworks/\\
de RDB, ML, NLP, ¿transfer learning?, ¿deep learning?,...\\
se de puede /diseñar/desarrollar/, ¿e implementar? una\\
/plataforma/framework/\\
/genérica/normalizada/universal/reutilizable/estandarizada/compatible\\
/con/en/para/ /casos de uso similares/diversos casos de uso/\\
(/reduciendo tiempo/ahorrando recursos/.) estas oración se sale del scope

LO NUEVO
Haciendo uso de las ciencias de la computación, las herramientas matemáticas de estadística y probabilidad, y métodos de aprendizaje autónomo se busca obtener información cuantitativa de textos provenientes de redes sociales, cadenas noticiosas y audio de programas de capacitación, para inferir posiciones, tendencias, comportamientos o razones de grupos sociales, considerando un ciclo de clasificación, estimación, detección y comprobación.

Particulares
Desarrollar un modelo de base de datos que permita la captura de categorías para un determinado problema, los elementos de identificación de cada categoría, el origen de la información y su correlación.
Construir una estructura de datos que capte la estimación o valores esperados para el procesamiento de textos
Elaborar un sistema de objetos para el soporte de los elementos de aprendizaje autónomo
Generar los elementos de captura de textos para su almacenamiento y procesamiento
Elaborar un modelo estadístico que permita comprobar las estimaciones a partir de los datos y en consecuencia realizar un ajuste en los parámetros usados para el aprendizaje autónomo
Producir los reportes con un análisis estadístico que faciliten la interpretación de resultados y den pauta para la obtención del conocimiento de interés.
