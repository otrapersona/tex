\documentclass[12pt,letterpaper]{article}
\usepackage[utf8]{inputenc}
\usepackage[spanish]{babel}
\author{author}
\title{title}
\usepackage{csquotes}
\usepackage{amsmath}
\usepackage{amssymb}
\usepackage{amsfonts}
\usepackage{mathtools}
\usepackage{multirow}
\usepackage{array}
\usepackage{tabularx}
\usepackage[dvipsnames]{xcolor}
%\usepackage[backend=biber,style=alphabetic,sorting=nyt]{biblatex} %\addbibresource{bibfile.bib}
\usepackage[backend=bibtex,style=alphabetic]{biblatex}
\bibliography{pro}
\usepackage{newtxmath}
\begin{document}
\section {Marco teórico}
\subsection {Correlación lineal}
Cuando se tiene una variable controlada $x$ y una dependiente $y$ tenemos el modelo lineal
\begin{equation}
Y=\beta_0+\beta_1x+\epsilon
\end{equation}
que implica entonces el modelo para análisis de rendimiento promedio:
\begin{equation}\label{eq:EQ1}
E(Y)=\beta_0+\beta_1x.
\end{equation}

Si la variable $x$ es un valor observado de una variable $X$, al establecerse una relación funcional y al basarse en \eqref{eq:EQ1}
se implica el modelo
\begin{equation}
E(Y|X=x)=\beta_0+\beta_1x
\end{equation}
que supone la esperanza condicional de $Y$ para un valor fijo de $X$ en una función lineal del valor $x$. Al suponer que la variable aleatoria vectorial $(X, Y)$ tiene una distribución normal bivariable con $E(X)=\mu_X, E(Y)=\mu_Y, V(X)=\sigma^2_X, V(Y)=\sigma^2_Y$, el coeficiente de correlación $\rho$ puede demostrar que
\begin{equation}
E(Y|X=x)=\beta_0+\beta_1x,\qquad donde\ \beta_1=\frac{\sigma_Y}{\sigma_X}\rho.
\end{equation}


Si $(X, Y)$ tiene una distribución normal bivariable, entonces la prueba de independencia es equivalente a probar si el coeficiente de correlación $\rho$ es igual a cero. Denotando con $(X_1, Y_1), (X_2, Y_2),\ldots , (X_n, Y_n)$ una muestra de aleatoria de distribución normal bivariante. El estimador de máxima probabilidad de $\rho$ está dado por el coeficiente de correlación muestral:
\begin{equation}
r=\frac{\sum_{i=1}^{n}(X_i-\overline{X})(Y_i-\overline{Y})}{\sqrt{\sum_{i=1}^{n}(X_i-\overline{X})^2\ \sum_{i=1}^{n}(Y_i-\overline{Y})^2}}.
\end{equation}

Puede expresarse $r$ en términos de cantidades conocidas:
\begin{equation}
r=\frac{S{xy}}{\sqrt{S{xx}S{yy}}}=\hat{\beta}\sqrt{\frac{S{xx}}{S{yy}}}.
\end{equation}

Cuando $(X, Y)$ tenga una distribución normal bivariable, se sabe que
\begin{equation}
E(Y|X=x)=\beta_0+\beta_1x,\qquad donde\ \beta_1=\frac{\sigma_Y}{\sigma_X}\rho.
\end{equation}

Pruebas en las que los conjuntos de hipótesis que contienen $\beta_1$, por ejemplo, $H_a\colon \beta_1 = 0$ contra $H_a\colon \beta_1 > 0$ $H_a\colon \beta_1 < 0$, así como $H_a\colon \beta_1 > 0$ contra $H_a\colon \beta_1 \neq 0$ pueden estar basadas en el estadístico
\begin{equation}
t=\frac{\hat{\beta_1}-0}{S/\sqrt{S{xx}}},
\end{equation}

Wackerly, Mendenhall III y Scheaffer consideran que \fullcite[``parecería lógico usar $r$ como estadístico de prueba para probar hipótesis más generales acerca de $\rho$, pero la distribución de probabilidad para $r$ es difícil de obtener."][]{wack09} Sin embargo, en muestras moderadamente grandes podemos probar la hipótesis $H_0\colon \rho_1=\rho_0$ con una prueba Z en la que
\begin{equation}
Z=\frac{(\frac{1}{2})\ln({\frac{1+r}{1-r})}-(\frac{1}{2})\ln(\frac{1+\rho}{1-\rho})}{\frac{1}{\sqrt{n-3}}}.
\end{equation}


Si $\alpha$ es la probabilidad deseada de cometer un error tipo I, la forma de la región de rechazo depende de la hipótesis alternativa. Las diversas alternativas de interés más frecuente y correspondientes regiones de rechazo son las siguientes:
\begin{equation} H_a\colon\rho>\rho0,\quad RR\colon z>z_\alpha \end{equation}
\begin{equation} H_a\colon\rho<\rho0,\quad RR\colon z<-z_\alpha \end{equation}
\begin{equation} H_a\colon\rho\neq\rho0,\quad RR\colon | z | >z_\alpha/2 \end{equation}

La suma de los cuadrados del error \emph{SSE}, es es una alternativa para medie la variación en valores que permanecen sin explicación después de usar las $x$ para ajustar el modelo de regresión lineal simple, la razón $SSE/S_{yy}$ la proporción de la variación total en las $y_i$ que este modelo no explica. El coeficiente de determinación se puede escribir como
\begin{equation}
r^2=(\frac{S_{xy}}{\sqrt{S_{xx}S_{yy}}})^2=(\frac{S_{xy}}{S_{xx}})(\frac{S_{xy}}{S_{yy}})=(\frac{\hat{\beta_1}S_{xy}}{S_{yy}})=\frac{S_{yy}-SEE}{S_{yy}}=1-\frac{SSE}{S_{yy}}.
\end{equation}



Podemos interpretar a $r^2$ como la proporción de la variación total en las $y_i$  que es explicada por una variable $x$ en un modelo de regresión lineal simple.

\newpage
\subsection {Modelo de base de datos relacional}

\begin{center}
\begin{tabular}{|| m{1.1cm} | m{1.1cm} || m{1.3cm} | m{1.3cm} || m{1.7cm} | m{1.7cm} ||}
\hline
\multicolumn{1}{|| c |}{código} & \multicolumn{1}{ c ||}{CHAR} & \multicolumn{1}{ c |}{fecha} & \multicolumn{1}{ c ||}{DATE} & \multicolumn{1}{ c |}{estado} & \multicolumn{1}{ c ||}{INTEGER}\\
\hline
\hline
\multicolumn{1}{|| c |}{MX01} & \multicolumn{1}{ c ||}{\ } & \multicolumn{1}{ c |}{2020-01-01} & \multicolumn{1}{ c ||}{\ } & \multicolumn{1}{ c |}{1} & \multicolumn{1}{ c ||}{\ }\\
\hline
\multicolumn{1}{|| c |}{MX02} & \multicolumn{1}{ c ||}{\ } & \multicolumn{1}{ c |}{2020-01-02} & \multicolumn{1}{ c ||}{\ } & \multicolumn{1}{ c |}{1} & \multicolumn{1}{ c ||}{\ }\\
\hline
\multicolumn{1}{|| c |}{MX03} & \multicolumn{1}{ c ||}{\ } & \multicolumn{1}{ c |}{2020-01-03} & \multicolumn{1}{ c ||}{\ } & \multicolumn{1}{ c |}{0} & \multicolumn{1}{ c ||}{\ }\\
\hline
\end{tabular}
\end{center}\cite{date12}
\newpage
\subsection {Probabilidad condicional}
La probabilidad condicional tiene las mismas características que la probabilidad, pero $P(\cdot|B)$ actualiza nuestra incertidumbre acerca de los eventos para reflejar la evidencia observada en $B$.

Si $A$ y $B$ son eventos con $P(B)>0$ entonces la \emph{probabilidad condicional} de $A$ dado $B$ denotado por $P(A|B)$, se define como
\begin{equation}
P(A|B)=\frac{P(A\cap B)}{P(B)}.
\end{equation}

Se denomina probabilidad a priori de $A$ a $P(A)$ y probabilidad a posteriori de $A$ a $P(A|B)$ y es importante mencionar que $P(A|B) \neq P(B|A)$.

La probabilidad condicional es la razón de dos probabilidades y sus consecuencias, la primera de ellas se obtiene moviendo el denominador en la definición al otro lado de la ecuación, para cada evento {A} y {B} con posibilidades positivas,
\begin{equation}\label{eq:EQ2}
P(A\cap B)=P(B)P(A|B)=P(A)P(B|A).
\end{equation}
se le conoce como teorema de la \emph{probabilidad de la intersección de dos eventos}. Aplicando repetidamente el teorema \eqref{eq:EQ2} aplicado a la intersección de $n$ eventos obtenemos el teorema de \emph{probabilidad de la intersección de $n$ eventos}. Para cualquier evento $A_1,\ldots,A_n$ con probabilidad $P(A_1,A_2,\ldots,A_{n-1})>0$,
\begin{equation}
P(A_1,A_2,\ldots,A_n)=P(A_1)P(A_2|A_1)P(A_3|A_1,A_2)\ldots P(A_n|A_1,\ldots,A_{n-1}).
\end{equation}

Otro teorema que relaciona a $P(A|B)$ con $P(B|A)$ es la regla \emph{regla de Bayes}:
\begin{equation}
P(A|B)=\frac{P(B|A)P(A)}{P(B)}
\end{equation}
que se origina directamente del teorema \eqref{eq:EQ2} y a su vez se origina directamente de la definición de probabilidad condicional, sin embargo, la regla de Bayes tiene importantes aplicaciones e implicaciones en probabilidad y estadística, ya que en ocasiones es más fácil encontrar $P(B|A)$ que $P(A|B)$ o viceversa. Otra forma de escribir la regla, es en términos de sus \colorbox{yellow}{odds COMO SE DICE EN ESPAÑOL, ?`cuota?}, las \colorbox{yellow}{odds} de un evento $A$ son
\begin{equation}
odds(A)=P(A)\frac{P(A)}{P(A^\text{c})}.
\end{equation}
Al tomar la expresión $P(A|B)$ y la dividimos entre la $P(A^\text{c}|B)$, ambas de la regla de Bayes llegamos a la \emph{forma de \colorbox{yellow}{odds} de la regla de Bayes} que para cualquier eventos $A$ y $B$ con posibilidades positivas, las \colorbox{yellow}{odds} de $A$ \colorbox{yellow}{after conditioning on} $B$ son
\begin{equation}
\frac{P(A|B)}{P(A^\text{c}|B)}=\frac{P(B|A)}{P(B|A^\text{c})}\frac{P(A)}{P(A^\text{c})}.
\end{equation}
En este caso las \colorbox{yellow}{odds} a posteriori $P(A|B)/P(A^\text{c}|B)$ son iguales a las \colorbox{yellow}{odds} a priori $P(A)/P(A^\text{c})$ por el factor $P(B|A)/P(B|A^\text{c})$ lo que se le conoce en estadística como \emph{función de verosimilitud}.

La ley de Bayes es usada en ocasiones en conjunto con la \emph{ley de probabilidad total}, que es es esencial para descomponer problemas complicados de probabilidad en problemas partes: Si $A_1,\ldots,A_n$ es una partición de una muestra del espacio $S$, con $P(A_i)>0$ para todo $i$, entonces
\begin{equation}
P(B)=\sum_{i=1}^{n}P(B|A_i)P(A_i).
\end{equation}
Prueba: como los $A_i$ forman una partición de $S$, podemos descomponer $B$ como
\begin{equation}
B=(B\cap A_1)\cup(B\cap A_2)\cup\ldots\cup(B\cap A_n).
\end{equation}
como las partes están disjuntas, podemos agregar sus posibilidades para obtener $P(B)$:
\begin{equation}
P(B)=P(B\cap A_1)+P(B\cap A_2)+\ldots+P(B\cap A_n).
\end{equation}
Aplicando el teorema \eqref{eq:EQ2} a cada $P(B\cap A_i)$ obtenemos
\begin{equation}
P(B)=P(B|A_1)P(A_1)+\ldots+P(B|A_n)P(A_n).
\end{equation}
La \emph{ley de probabilidad total} dice que para obtener la probabilidad incondicional de $B$, tenemos que dividir el espacio muestra en cortes disjuntos $A_i$, encontrar la probabilidad condicional de $B$ en cada corte, después tomar la suma pesada de las probabilidades condicionales, donde los pesos son las probabilidades $P(A_i)$.
\subsection {Variables aleatorias y sus distribuciones}
Dado un experimento con una muestra en espacio $S$, una \emph{variable aleatoria} es una función del mismo espacio $S$ para los números reales $\mathbb{R}$. Una variable aleatoria $X$ asigna el valor numérico $X(s)$ a cada resultado posible $s$ del experimento. La aleatoriedad viene del hecho que tenemos un experimento aleatorio (con probabilidades descritas por la función de probabilidad $P$). Las variables aleatorias simplifican la notación y expanden la habilidad de cuantificar y resumir resultados de experimentos.

Se dice que una variable $X$ es discreta cuando si hay una lista finita de valores $a_,a_2,\ldots,a_n$ o un una lista infinita de valores $a_,a_2,\ldots$ de tal forma que $P(X=a_j$ para algún $j)=1$. Si $X$ es una variable aleatoria discreta, entonces el grupo infinito o contable de valores $x$ tal que $P(X=x)$ se llama \emph{soporte} de $X$. En contraste una variable aleatoria continua puede tomar cualquier valor real en un intervalo.

La forma más natural de expresar la distribución de variables aleatorias discretas es la \emph{función de probabilidad}\cite{blitz19} \colorbox{yellow}{(PMF, por sus siglas en inglés)} que, para una $X$ discreta, es la función $p_X$ dada por $p_X(x)=P(X=x)$. El teorema de \emph{funciones de probabilidad válidas} dice que cuando $X$ es una variable aleatoria con soporte $x1,x2,\ldots$, la función de probabilidad $p_X$ de $x$ debe satisfacer los siguiente criterios:\begin{itemize}
    \item No negativo $p_X (x) > 0$ si $x=x_j$ para un $j$, y $p_X(x)=0$, de otra forma;
    \item Suma 1: $\sum_{j=1}^{\infty}p_X(x_j)=1$.
\end{itemize}
el primer criterio es verdadero porque la probabilidad es no negativa, el segundo es verdadero ya que $X$ debe tomar \emph{algún} valor, y los eventos ${X=xj}$ están disjuntos, entonces
\begin{equation}
\sum_{j=1}^{\infty}P(X=x_j)=P\bigg(\bigcup_{j=1}^{\infty}\{X=x_j\}\bigg)=P(X=x_1\ \text{ó}\ X=x_2\ \text{ó}\ \ldots)=1.
\end{equation}
\subsubsection {Distribución de Bernoulli}
Una variable aleatoria tiene la \emph{distribución de Bernoulli} con un parámetro $p$ si $P(X=1)=p$ y $P(X=0=1-p)$, cuando $0<p<1$. Se escribe como $X \sim Bern(p)$






















\newpage
Maybe\cite{blitz19}
\newpage
\printbibliography[heading=bibintoc,title={Fuentes}]
\end{document}
