\documentclass[letterpaper]{article}
\usepackage[utf8]{inputenc}
\usepackage[spanish]{babel}
\author{Victor Matus}
\title{Protocolo}
\date{\today}
%>>> Mate, símbolos
\usepackage{amsmath}
\usepackage{amssymb}
\usepackage{amsfonts}
\usepackage{mathtools}
\usepackage{newtxmath} % fuente?
\usepackage{array}
\usepackage{multirow} % tablas
\usepackage{tabularx} % tablas
%<<< Mate, símbolos
\usepackage[dvipsnames]{xcolor} % marcatextos, se puede ir al final?
%>>> Bibliografía
\usepackage{csquotes} % citar
%\usepackage[backend=biber,style=alphabetic,sorting=nyt]{biblatex} % OVERLEAF
%\addbibresource{bibfile.bib} % OVERLEAF
\usepackage[backend=bibtex,style=alphabetic]{biblatex} % genérico no-overleaf
\bibliography{pro} % genérico no-overleaf
%>>>  Bibliografía
%>>> Imágenes
\usepackage{graphicx} % insertar
\usepackage{float} % ubicación flotante
%>>> Imágenes
%>>> Clickable hypers
\usepackage{hyperref}
\hypersetup{
    hyperindex=true,
    linktocpage=false,
    colorlinks=true,
    allcolors=black,
    bookmarksnumbered=true,
    unicode=true,
}
%<<< Clickable hypers
\begin{document}
\maketitle
%\begin{abstract}
%    Computer = good.
%Observe la figura \ref{fig:1}.
%\begin{figure}[h]\centering
%    \caption{A computer}\label{fig:1}
%    \includegraphics[width=2cm]{computer}
%\end{figure}
%\end{abstract}
%\pagebreak
\tableofcontents
\subsubsection{to-do}\label{ch:to-do}
\input{to-do}
\section {Marco teórico}
\subsection {Álgebra relacional}\label{ch:algebra}
Los cimientos del modelo relacional son el \emph{álgebra relacional}, las operaciones del álgebra producen nuevas relaciones, que pueden manipularse también por medio de operaciones del álgebra mismo. Una secuencia de operaciones de álgebra relacional forma una \emph{expresión de álgebra relacional} cuyo resultado es una relación que representa el resultado de una consulta (o solicitud de consulta) de base de datos. Estas operaciones pueden clasificar en dos grupos, operaciones de conjuntos*\notad{(así se llaman? checar libro en l'espagnol)``set operations''} de la teoría de conjuntos matemáticos refiere a la operaciones \emph{UNION}, \emph{INTERSECTION}, \emph{SET DIFFERENCE} (\emph{MINUS}) y \emph{CARTESIAN PRODUCT} (\emph{CROSS PRODUCT}). El otro grupo consiste en operaciones específicas para bases de datos relacionales: \emph{JOIN}, \emph{SELECT} y \emph{PROJECT}. Estas dos últimas, por operar con una sola relación, son también conocidas como \emph{operaciones unarias}.

SELECT elige un subconjunto de tuplas de una relación que satisfacen la condición de selección
\begin{equation}
\sigma_\text{\textless condición de selección\textgreater}{\text{(R)}}
\end{equation}
donde $\sigma$ denota el operador SELECT, y la condición de selección es una expresión booleana especificada en los atributos de la relación $R$.

PROJECT produce una nueva relación de atributos y tuplas*\notaihi{aquí ya hablo de tuplas que se explican en el modelo relacional} \notadhi{tiene más sentido hablar de él primero, así lo hacen los libros} duplicadas
\begin{equation}
\pi_\text{\textless lista de atributos\textgreater}{\text{(R)}}
\end{equation}
donde $\pi$ es el denota el operador PROJECT y la lista de atributos es la sublista de atributos deseados de la relación $R$.

Las operaciones con dos relaciones reciben el nombre de \emph{operaciones binarias}. Si las relaciones $R(A_1,A_2,\ldots,A_n)$ y $S(B_1,B_2,\ldots,B_n)$ son compatibles de unión (tienen el mismo grado $n$ y $dom(A_i)=dom(B_i)$ para $1\leq i\leq n$) podemos usarlas para definir las siguientes operaciones binarias:
\begin{itemize}
\item UNION: El resultado de esta operación se denota $R \cup S$, es una relación que incluye todas las tuplas que están en $R$, $S$ o ambos, se eliminan duplicados.
\item INTERSECTION: El resultado de esta operación se denota $R \cap S$, es una relación que incluye todas las tuplas que están en ambos $R$ y $S$.
\item SET DIFFERENCE: El resultado de esta operación se denota $R - S$, es una relación que incluye todas las tuplas que están en ambos $R$ pero no en $S$.
\end{itemize}

El operador CARTESIAN PRODUCT, denotado $R \times S$ es la operación binaria que no requiere compatibilidad de unión y produce un nuevo elemento al combinar cada miembro (tupla) de cada relación conjunto) con cada otro miembro de la otra relación. El resultado de $R(A_1,A_2,\ldots,A_n)$ y $S(B_1,B_2,\ldots,B_m)$ es una relación $Q$ con atributos $Q(A_1,A_2,\ldots,A_n,B_1,B_2,\ldots,B_m)$ (en ese orden) de grado $n+m$. El resultado $Q$ tiene una tupla por cada combinación de tuplas de $R$ y $S$.

JOIN $\Join$, es el operador utilizado para combinar tuplas relacionadas de dos relaciones en una sola tupla. Pertinente para procesar relaciones entre relaciones. Si tenemos dos relaciones $R(A_1,A_2,\ldots,A_n)$ y $S(B_1,B_2,\ldots,B_m)$ podemos escribir la operación JOIN como
\begin{equation}
R\Join_\text{\textless condiciones de unión\textgreater}S
\end{equation}
el resultado de la unión es la relación $Q$ con atributos $Q(A_1,A_2,\ldots,A_n,B_1,B_2,\ldots,B_m)$ (en ese orden). El resultado $Q$ tiene una tupla por cada combinación de tuplas de $R$ y $S$ que satisfacen las condiciones de unión.
\subsection {Modelo de base de datos relacional}
El modelo relacional de base de datos, según Date \cite{date12} consiste en cinco componentes:
\begin{enumerate}
    \item Una colección de tipos escalares, pueden ser definidos por el sistema (INTEGER, CHAR, BOOLEAN, etc.) o por el usuario.
    \item Un generador de tipos de relaciones y un intérprete para las relaciones mismas.
    \item Estructuras para definir variables relacionales de los tipos generados.
    \item Un operador para asignar valores de relación a dichas variables.
    \item Una colección relacionalmente completa para obtener valores relacionales de otros valores relacionales mediante operadores relacionales genéricos (Álgebra relacional \ref{subsec:algebra}).
\end{enumerate}
Es importante comenzar definiendo los tipos, ya que las relaciones se definen sobre ellos, según Date, los tipos son ``en esencia un conjunto finito de valores nombrado-todos los valores posibles de alguna categoría específica, por ejemplo, todos los números enteros posibles, todos los caracteres string posibles, todos los teléfonos de proveedores posibles, todos los documentos XML posibles, todas las relaciones con cierta cabecera posibles(y así sucesivamente)'' \cite{date12}.

Cada atributo de cada relación es definido como de un tipo. Los atributos son pares ordenados de combinaciones atributo-nombre/tipo-nombre y una tupla es un par ordenado de atributos.
El modelo relacional también soporta varios tipos de llaves, que poseen las propiedades de unicidad, ninguna contiene dos tuplas distintas con el mismo valor e irreductibilidad, ningún subconjunto suyo es tiene unicidad. La llave foránea (\emph{FK}) es una combinación o set se atributos FK en una relación $r2$ tal que se requiere que cada valor FK sea igual a algún valor de alguna llave K en alguna relación $r1$ ($r1$ y $r2$ no son necesariamente distintos).

Una restricción de integridad (\emph{constraint}) es una expresión booleana que debe evaluarse como verdadera. Los constraints de tipo definen los valores que constituyen un tipo dado, mientras que los constraints de base de datos limitan los valores que pueden aparecer en cierta base de datos. Las bases de datos suelen tener múltiples constraints específicos, expresados en términos de sus relaciones, sin embargo, el modelo relacional incluye dos constraints genéricos, que aplican a cada base de datos:
\begin{itemize}
    \item Regla de integridad de identidad: Las llaves primarias no pueden ser nulas (\emph{null}).
    \item Regla de integridad de referencia: No debe haber valores FK sin relación (si $B$ referencia a $A$, $A$ debe existir).
\end{itemize}
\begin{table}[H]\begin{tabular}{rc|c|c|l}
\cline{2-7}
\multicolumn{1}{r}{$Atributos\begin{cases}\ \end{cases}$}&\multicolumn{1}{|c|}{código}&\multicolumn{1}{c|}{fecha}&\multicolumn{1}{c|}{estado}&\multicolumn{1}{ l }{}\\
\cline{2-7}
\multirow{3}{*}{$Tuplas\begin{cases}\\\\\end{cases}$}&\multicolumn{1}{| c }{MX01}& 29-07-99 &0&\multicolumn{1}{ l }{}\\
\cline{2-7}
&\multicolumn{1}{| c }{MX02}&30-07-99&0&\multicolumn{1}{ l }{}\\
\cline{2-7}
&\multicolumn{1}{| c }{MX03}&31-07-99&0&\multicolumn{1}{ l }{}\\
\cline{2-7}
\end{tabular}\caption{Representación de una tabla de una base en datos, la fila superior muestra las combinaciones de atributo-nombre/tipo-nombre para algunos atributos en la tabla y \emph{cada una} de las filas siguientes es una tupla.}\label{table:tupla}\end{table}
\subsection {Procesamiento de lenguaje natural}
El \emph{procesamiento de lenguaje natural}, es el conjunto de métodos para hacer accesible el lenguaje humano a las computadoras\cite{eise19}.*\notaihi{como que falta algo aquí?} Existen dos enfoques en lo que debe ser su tarea central: 
\begin{itemize}
    \item Entrenar sistemas de extremo a extremo*\notadhi{end-to-end, de principio a fin?} que transmuten texto sin procesar a cualquier estructura deseada.
    \item Transformar texto en una pila de estructuras lingüísticas de uso general que en teoría deben poder soportar cualquier aplicación.
\end{itemize}
Dos de los módulos básicos de NLP son \emph{búsqueda} y \emph{aprendizaje} con los que se puede resolver muchos problemas que podemos describir en la siguiente forma matemática
\begin{equation}
\begin{matrix}
\hat{y}=argmax\Psi(x,y;0),\\
y\in Y(x)
\end{matrix}
\end{equation}
donde,
\begin{itemize}
    \item $x$ es la entrada, un elemento de un conjunto $X$.
    \item $y$ es el resultado, un elemento de un conjunto $Y$.
    \item $\Psi$ es una función de puntuación (también conocida como \emph{modelo}), que va desde el conjunto $X\times Y$ hasta los números reales.
    \item $\emptyset$ es el vector de parámetros para $\Psi$.
    \item $\hat{y}$ es el resultado previsto, que es elegido para maximizar la función de puntuación.
\end{itemize}
El módulo de búsqueda se encarga de computar el $argmax$ de la función $\Psi$, es decir, encuentra el resultado $\hat{y}$ con la mejor puntuación con respecto a la entrada $x$. El módulo de aprendizaje encuentra los parámetros $\theta$ por medio del procesamiento de grandes conjuntos de datos de ejemplos etiquetados ${\{(x^i,y^i)\}}_{i=1}^{N}$.
\subsection {Correlación lineal}
\input{correlacion}
\subsection {Probabilidad condicional}
\input{probcond}
\subsection {Variables aleatorias y sus distribuciones}
\subsubsection {Distribución de Bernoulli y binominal}
Una variable aleatoria tiene la \emph{distribución de Bernoulli} con un parámetro $p$ si $P(X=1)=p$ y $P(X=0=1-p)$, cuando $0<p<1$. Se escribe como $X \sim Bern(p)$, el símbolo $\sim$ significa ``distribuido como'' y la probabilidad $p$ es el \emph{parámetro}, que determina qué distribución de Bernoulli específica tenemos.

Supóngase que se realizan $n$ ensayos Bernoulli independientes, cada uno con probabilidad $p$ de éxito. $X$ sea el número de éxitos, la distribución $X$ se llama \emph{distribución binominal} con parámetros $n$ y $p$; se escribe $X \sim Bin(p,n)$.
$Bern(p)$ es la misma distribución que $Bin(1,p)$. Bernoulli es un caso especial de binominal, si $x \sim Bin(1,p)$, entonces la función de probabilidad de $X$ es
\begin{equation}
P(X=k)=\binom{n}{k}p^k(1-p)^{n-k}
\end{equation}
para $k=0,1,\ldots,n$ (y por otra parte $P(X=k)=0$).
\subsubsection {Distribución de hipergeométrica}
Si $X \sim HGeom(w,b,n)$, entonces la función de probabilidad de X es
\begin{equation}
P(X=k)=\frac{\binom{w}{k}\binom{b}{n-k}}{\binom{w+b}{n}},
\end{equation}
para enteros $k$ satisfaciendo $0\leq k\leq w$ y $0\leq n-k\leq b$, y $P(X=k)=0$. La estructura esencial de la distribución hipergeométrica se basa en que objetos en su población están clasificados usando dos tipos de etiquetas, al menos una de estas siendo asignada al azar.
Las distribuciones $HGeom(w,b,n)$ y $HGeom(n,w+b-n,1)$ son idénticas si $X$ y $Y$ tienen la misma distribución, podemos demostrarlo algebraicamente:
\begin{equation}
P(X=k)=\frac{\binom{w}{k}\binom{b}{n-k}}{\binom{w+b}{n}}=\frac{w!b!n!(w+b-n)!}{k!(w+b)!(w-k)!(n-k)!(b-n+k)!}
\end{equation}
\begin{equation}
P(X=k)=\frac{\binom{n}{k}\binom{w+b-n}{w-k}}{\binom{w+b}{w}}=\frac{w!b!n!(w+b-n)!}{k!(w+b)!(w-k)!(n-k)!(b-n+k)!}.
\end{equation}%if intro for bernolli and hyper neede, 3.4.6 has a bit a few useful lines
\subsubsection {Distribución uniforme discreta}
Teniendo $C$, un conjunto finito no vacío de números, se elige un número uniformemente al azar (o sea que todos los números tienen la misma posibilidad de ser elegidos), llámese $X$. Entonces se dice que $X$ una \emph{distribución uniforme discreta} con el parámetro $C$. Se dice entonces que la función de probabilidad de $X \sim DUNif(C)$ (la distribución uniforme discreta de $X$) es
\begin{equation}
P(X=x)=\frac{1}{|C|}
\end{equation}
para $x \in C$ (de lo contrario $0$) ya que la función de probabilidad debe sumar 1.
\subsubsection {Función de distribución acumulada}
Esta función describe la distribución de todas las variables aleatorias (a diferencia de la función de probabilidad que sólo se aplica a las discretas). La \emph{función de distribución acumulada} de una variable aleatoria $X$ es la función $F_X$ dada por $F_X(x)=P(X\leq x)$ y tiene las siguientes propiedades:
\begin{itemize}
    \item Incrementos: Si $x_1\leq x_2$, then $F(x_1)\leq F(x_2)$.
    \item Continua por la derecha: Es continua por la posibilidad de tener saltos. Cuando hay saltos es continua por la derecha, es decir, por cada $a$ se tiene
    \begin{equation}
    F(a)=\lim_{c\to a^+}F(x).
    \end{equation}
    \item Convergencia de $0$ y $1$ en los límites
    \begin{equation}
    \lim_{x\to \infty}F(x)=0\ \ \text{y}\ \lim_{x\to \infty}F(x)=1.
    \end{equation}
\end{itemize}
\subsection {Valor esperado}
\subsection {distribuciones}\label{subsec:dd}
\subsubsection {Binominal geométrica y negativa}
Distribución geométrica: Se tiene una secuencia de ensayos independientes Bernoulli, cada uno con la misma probabilidad de éxito $p\in(0,1)$, con ensayos realizados hasta que se alcanza el éxito. $X$ es el número de \emph{fallas} antes de la primera prueba exitosa por lo que $X$ tiene una \emph{distribución geométrica} con un parámetro $p$; denotado $X\sim Geom(p)$. Con esto podemos llegar a los teoremas de \emph{distribución geométrica de la función de probabilidad}, cuando $X\sim Geom(p)$, entonces la función de probabilidad de $X$ será
\begin{equation}
P(X=k)=q^kp
\end{equation}
para $k=1,2,\ldots,$ cuando $q=1-p$; y el teoremas de \emph{distribución geométrica de la función de distribución acumulativa}, cuando $X\sim Geom(p)$, entonces la función de distribución acumulativa de $X$ será
    \begin{equation}
    F(x)=
    \begin{cases}
    1-q^{\lfloor x\rfloor+1}, \text{ si } x\geq 0;\\
    0, \text{ si }x < 0,
    \end{cases}
    \end{equation}
cuando $q=1-q$ y $\lfloor x\rfloor$ es el mayor entero y menor o igual a $x$.

El valor esperado geométrico de $X\sim Geom(p)$ es
\begin{equation}
E(X)=\sum_{k=0}^{\infty}kq^kp,
\end{equation}
cuando $q=1-p$. Aunque esta no es una serie geométrica, podemos llegar a ello
\begin{equation}\begin{matrix}
\sum_{k=0}^{\infty}q^k=\frac{1}{1-q}\\
\\
\sum_{k=0}^{\infty}kq^{k-1}=\frac{1}{{1-q}^2},
\end{matrix}
\end{equation}
finalmente multiplicamos ambos lados por $pq$, recuperando la suma original que queríamos encontrar
\begin{equation}
E(X)=\sum_{k=0}^{\infty}kq^kp=pq\sum_{k=0}^{\infty}kq^{k-1}=pq\frac{1}{{(1-q)}^2}=\frac{q}{p}.
\end{equation}
Primer valor esperado de éxito \emph{FS}, podemos definir a $Y\sim FS(p)$ como $Y=X+1$ donde $X\sim Geom(p)$, por lo que tenemos
\begin{equation}
E(Y)=E(X+1)=\frac{q}{p}+1=\frac{1}{p}.
\end{equation}

Las \emph{distribuciones binominales negativas} generalizan la distribución geométrica en lugar de esperar por un éxito, podemos esperar por cualquier número predeterminado $r$ de éxitos. En una secuencia de ensayos independientes Bernoulli con probabilidad de éxito $p$, si $X$ es el número de \emph{fallas} antes del éxito número $r$, entonces se dice que $X$tiene una distribución binominal negativa con parámetros $r$ y $p$, denotado $X\sim NBin(r,p)$.

La distribución binominal cuenta el número de éxitos en un número fijo de ensayos, mientras que la binominal negativa cuenta el número de fallas hasta alcanzar cierto número de éxitos. Si $X\sim NBin(r,p)$, entonces la función de probabilidad de $X$ es
\begin{equation}
P(X=n)=\binom{n+r-1}{r-1}p^rq^n
\end{equation}
para $n=0,1,2\ldots,$ donde $q=1=p$.
%\subsubsection{LOTUS}
%La \emph{ley del estadista inconsciente} (\emph{LOTUS}, por sus siglas en inglés) permite calcular $E(g(X))$ directamente usando la distribución de $X$, sin tener que encontrar la distribución de $g(X)$ primero: si $X$ es una variable discreta y $g$ es una función de $\mathbb{R}$ a $\mathbb{R}$, entonces
%\begin{equation}
%E(g(X))=\sum_{x}g(x)P(X=x),
%\end{equation}
%donde la suma se toma de todos los valores posibles de $X$. El valor esperado de $g(X)$ puede ser escrito en forma no agrupado como
%\begin{equation}
%E(g(X))=\sum_{s}g(X(s))p(\{s\}).
%\end{equation}
%\subsubsection{Varianza, agregarlas a la secc que pertenecen}
%Varianza de la geométrica (agregarla a la geom después de editar todo y hacerlo más breve)
%\begin{equation}
%Var(X)=E(X^2)-{(EX)}^2=\frac{q(1+q)}{p^2}-{(\frac{q}{p})}^2=\frac{q}{p^2}
%\end{equation}
%Varianza de la geométrica (lo mismo)
%\begin{equation}
%Var(X)=E(X^2)-{(EX)}^2=(n(n-1)p^2+np)-(np)^2=np(1-p).
%\end{equation}
%Una variable aleatoria $X$ tiene \emph{distribución de Poisson} (denotada $X\sim Pois(\lambda)$) con el parámetro $\lambda$, cuando $\lambda > 0$ si la PMF de $x$ es
%\begin{equation}
%P(X=k)=\frac{e^{-\lambda}\lambda^k}{k!}, k=0,1,2,\ldots.
%\end{equation}
%Varianza de la distribución de Poisson es
%\begin{equation}
%Var(X)=E(X^2)-{(EX)}^2=\lambda(1+\lambda)-\lambda^2=\lambda
%\end{equation}






\section {Objetivos}
Comenzar desde cero cada proyecto es ineficiente. (esto es temporal)
\subsection {General}
Diseñar y desarrollar una\\
plataforma/framework de ML para NLP\\
reutilizable y/o de uso general/ ¿para casos de uso similares?
\subsection {Particulares}
Revisé algunas tesis para darme la idea, necesito saber un poco más del tema para desarrollar esta parte, creo que sería algo así:
\begin{enumerate}
    \item Investigación
    \item Análisis info
    \item Diseño/desarrollo
    \item Comprobación
\end{enumerate}
\subsection {Hipótesis}
Al integrar\\
/tecnologías/técnicas/modelos/librerías/frameworks/\\
de RDB, ML, NLP, ¿transfer learning?, ¿deep learning?,...\\
se de puede /diseñar/desarrollar/, ¿e implementar? una\\
/plataforma/framework/\\
/genérica/normalizada/universal/reutilizable/estandarizada/compatible\\
/con/en/para/ /casos de uso similares/diversos casos de uso/\\
(/reduciendo tiempo/ahorrando recursos/.) estas oración se sale del scope

\section {Metas}
Contribuir por medio del diseño y desarrollo de la\\
/plataforma/framework/ de ML para NLP\\
que será utilizable en variedad de casos reales /con características similares./
\section {Metodologías}
%\newpage
%\renewcommand{\listfigurename}{Índice de figuras (imágenes, fotos)} %cambiaelnombre
%\listoffigures
%\renewcommand{\listtablename}{Índice de cuadros (plots, gráficas, charts)} %cambiaelnombre
%\listoftables
\newpage
\section {Referencias}
\printbibliography[heading=none]
\end{document}
