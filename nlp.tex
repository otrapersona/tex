El \emph{procesamiento de lenguaje natural}, es el conjunto de métodos para hacer accesible el lenguaje humano a las computadoras\cite{eise19}.*\notai{como que falta algo aquí?} Existen dos enfoques en lo que debe ser su tarea central: 
\begin{itemize}
    \item Entrenar sistemas de extremo a extremo*\notaihi{end-to-end, de principio a fin?} que transmuten texto sin procesar a cualquier estructura deseada.
    \item Transformar texto en una pila de estructuras lingüísticas de uso general que en teoría deben poder soportar cualquier aplicación.
\end{itemize}
Dos de los módulos básicos de NLP son \emph{búsqueda} y \emph{aprendizaje} con los que se puede resolver muchos problemas que podemos describir en la siguiente forma matemática
\begin{equation}
\begin{matrix}
\hat{y}=argmax\Psi(x,y;0),\\
y\in Y(x)
\end{matrix}
\end{equation}
donde,
\begin{itemize}
    \item $x$ es la entrada, un elemento de un conjunto $X$.
    \item $y$ es el resultado, un elemento de un conjunto $Y$.
    \item $\Psi$ es una función de puntuación (también conocida como \emph{modelo}), que va desde el conjunto $X\times Y$ hasta los números reales.
    \item $\emptyset$ es el vector de parámetros para $\Psi$.
    \item $\hat{y}$ es el resultado previsto, que es elegido para maximizar la función de puntuación.
\end{itemize}
El módulo de búsqueda se encarga de computar el $argmax$ de la función $\Psi$, es decir, encuentra el resultado $\hat{y}$ con la mejor puntuación con respecto a la entrada $x$. El módulo de aprendizaje encuentra los parámetros $\theta$ por medio del procesamiento de grandes conjuntos de datos de ejemplos etiquetados ${\{(x^i,y^i)\}}_{i=1}^{N}$.