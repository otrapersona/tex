\section {Metodología científica}
\ \\
1 Recopilación de información en forma aislada e integrada con aplicaciones relacionadas al proyecto.\\
2 Construir los modelos para la obtención de la base de datos.\\
3 Obtener el conjunto de palabras correlacionadas con la palabra clave para realizar una búsqueda más extensa en los textos y estimar los modelos de regresión lineal y/o regresión múltiple con las palabras seleccionadas.\\
4 Realizar una búsqueda correlacionada complementaria.\\
5 A partir de los nuevos datos se obtienen otros modelos de regresión lineal y se realizan pruebas de hipótesis para comparar los parámetros $\beta_0$ y $\beta_1$ del modelo anterior. El objetivo de este proceso es obtener los mejores estimadores (con medidas de dispersión optimizadas).\\
6 Construcción de elementos gráficos (distribuciones y rectas de regresión lineal).\\
7 Comunicación de resultados vía publicación y/o congreso en eventos especializados.