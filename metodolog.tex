\section {Metodología científica}
\begin{enumerate}
\item Recopilación de información en forma aislada e integrada con aplicaciones relacionadas al proyecto.
\item Construir los modelos para la obtención de la base de datos.
\item Obtener el conjunto de palabras correlacionadas con la palabra clave para realizar una búsqueda más extensa en los textos y estimar los modelos de regresión lineal y/o regresión múltiple con las palabras seleccionadas.
\item Realizar una búsqueda correlacionada complementaria.
\item A partir de los nuevos datos se obtienen otros modelos de regresión lineal y se realizan pruebas de hipótesis para comparar los parámetros $\beta_0$ y $\beta_1$ del modelo anterior. El objetivo de este proceso es obtener los mejores estimadores (con medidas de dispersión optimizadas).
\item Construcción de elementos gráficos (distribuciones y rectas de regresión lineal).
\item Comunicación de resultados vía publicación y/o congreso en eventos especializados.
\end{enumerate}