\section {Metodología científica}
\# \# \# \# \# \# \# \# \# \# \# \# \# \# \# \# \# \# \# \# \# \# \# \# \# \# \# \# \# \# \# \# \# \# \# \# \# \# \# \# \# \# \# \# \# \# \# \# \# \# \# \# \# \# \# \# \# \# \# \# \# \# \# \#\\\\
ejemplo uam:\\
http://tesiuami.izt.uam.mx/uam/aspuam/presentatesis.php?recno=19133\&docs=UAMI19133.pdf\\
1. Revision y clasificacion de la literatura sobre trabajos relacionados.\\
2. Elaboracion de hipotesis y diseno de una estrategia de distribucion de contenido.\\
3. Desarrollo de un protocolo con base en la estrategia disenada.\\
4. Validacion del protocolo mediante simulacion.\\
5. Elaboracion de conclusiones.\\
6. Comunicacion idonea de resultados.\\
\# \# \# \# \# \# \# \# \# \# \# \# \# \# \# \# \# \# \# \# \# \# \# \# \# \# \# \# \# \# \# \# \# \# \# \# \# \# \# \# \# \# \# \# \# \# \# \# \# \# \# \# \# \# \# \# \# \# \# \# \# \# \# \#\\\\
ejemplo 2\\
1.3.Metodología de investigaciónA lo largo de esta sección, mostraremos el proceso de investigación que hemos seguido para desarrollar este estudio, el cual se compone de las siguientes etapas (véase figura 1.3)\\
Formulación del proyecto de investigación. Durante esta etapa se define la problemática aresolver y se plantean los objetivos a cumplir, además de establecer una planeación de lasactividades a realizar a lo largo del estudio con base en una correcta delimitación del tema ainvestigar. Adicionalmente, se construye una hipótesis, la cual será sometida a un proceso deaceptación o refutación\\
Revisión sistemática de la literatura. A través de esta actividad, se estudian las diferentestécnicas que utilizan un modelo de mitigación de falsos positivos utilizando el enfoque deaprendizaje maquinal. También, se identifican y se analizan las principales etapas que componena cada técnica (y que mejor se adapten al desarrollo de nuestra propuesta), las herramientas desoftware utilizadas, los casos de estudio explorados y las medidas de desempeño encargadas deevaluar dichas técnicas\\
Diseño de la propuesta. Con base en el conocimiento adquirido en la etapa anterior, se diseñauna propuesta cuyo objetivo es aumentar el número de defectos relevantes descubiertos antes dellegar a la fase de pruebas dentro del PDS. Dicha propuesta se basa en las diferentes etapas quelos autores utilizan en la creación de sus modelos y en las diversas propiedades de sus AAIT1,además de incorporar nuevas características. La metodología training and testing [22] es parteesencial en el desarrollo de esta etapa.\\
Evaluación de la propuesta. Una vez que la propuesta ha sido diseñada, se vuelve necesarioimplementarla con el fin de realizar una evaluación de la misma y obtener una serie de resultadosque nos brinden información acerca del desempeño de todos los modelos generados y en general,del comportamiento de nuestra AAIT. Durante esta etapa, se genera, se identifica, se analiza y seselecciona la información necesaria para la construcción y evaluación de modelos de clasificaciónde alertas basados en diferentes algoritmos de aprendizaje maquinal. Todas estas actividades sonla parte experimental de la propuesta.\\
Análisis de resultados. Con base en la selección de los mejores modelos (es decir, aquellos quecuentan con el mayor poder predictivo), se realiza un análisis comparativo del desempeñologrado por el conjunto de modelos propios o internos al proyecto y el conjunto de modelosexternos al mismo. Posteriormente, se estudia la fiabilidad de incorporar una técnica como lanuestra dentro del PDS; las ventajas, desventajas e implicaciones de la propuesta son discutidas alo largo de esta etapa.\# \# \# \# \# \# \# \# \# \# \# \# \# \# \# \# \# \# \# \# \# \# \# \# \# \# \# \# \# \# \# \# \# \# \# \# \# \# \# \# \# \# \# \# \# \# \# \# \# \# \# \# \# \# \# \# \# \# \# \# \# \# \# \#\\\\
OTRO\\
En este trabajo de tesis el problema del portafolio de inversion se presento a traves delmodelo de media-varianza, se emplearon restricciones de no negatividad, para prohibir so-luciones con ”inversiones negativas”, y restricciones que garanticen la inversion de todo elcapital  disponible,  mientras  que  mediante  la  funcion  objetivo  se  busco  un  equilibrio  entreriesgo y retorno mediante una suma ponderada. Para la solucion de este modelo se imple-mento la tecnica metaheurıstica, Optimizacion por Enjambre de Partıculas (PSO), que hareportado muy buenos resultados para la solucion de este tipo de problemas. Asimismo, sepropuso el uso de una tecnica basada en PSO llamada PSO-3P, la cual se caracteriza pordividir en tres fases el proceso de busqueda original de PSO. Para poder evaluar el desempenode los algoritmos desarrollados, se consideraron 5 instancias reportadas en la literatura espe-cializada. Los resultados obtenidos fueron evaluados estadısticamente y comparados con los reportados por Chang et al., quienes emplearon algoritmos inspirados en Recocido Simulado,Busqueda Tabu y Algoritmos Geneticos.\\
El  resto  del  trabajo  se  dividen  de  la  siguiente  forma.  En  el  capıtulo  cuatro  se  explicael funcionamiento de las estrategias de optimizacion PSO y PSO-3P, y se presenta el pseu-docodigo  de  cada  una  de  ellas.  En  el  capıtulo  cinco  se  incluye  la  revision  de  la  literaturaespecializada.  En  este  espacio  se  realiza  una  introduccion  a  la  teorıa  del  portafolio  de  in-version, se presenta la formulacion del modelo de media-varianza, y se hace mencion de unmodelo de media-varianza con restricciones de cardinalidad. Asimismo se describen algunosde  los  modelos  que  se  han  propuesto  para  la  construccion  de  los  portafolios  de  inversionpartiendo del modelo de media-varianza, y se presenta la revision de algunos de los meto-dos empleados para la solucion de estos modelos. Para finalizar el capıtulo cinco se describedetalladamente el trabajo de Chang et al., que sirvio como referencia para este trabajo detesis. En el capıtulo seis se describe la adaptacion de los metodos de PSO y PSO-3P al pro-blema del portafolio de inversion. En el capıtulo siete se muestran los resultados, y analisisestadısticos, de los experimentos computacionales realizados sobre las instancias reportadasen el trabajo de Chang et al. El ultimo capıtulo, capıtulo ocho, se presentan las conclusionesy las recomendaciones para trabajo futuro.\\
\# \# \# \# \# \# \# \# \# \# \# \# \# \# \# \# \# \# \# \# \# \# \# \# \# \# \# \# \# \# \# \# \# \# \# \# \# \# \# \# \# \# \# \# \# \# \# \# \# \# \# \# \# \# \# \# \# \# \# \# \# \# \# \#\\\\
1.Análisis del problema planteado y de los requerimientos a solucionar.\\
2.Realizar un bosquejo del sistema para satisfacer los requerimientos indicados.\\
3.Diseñar el diagrama E-R correspondiente a la gestión de datos identificados.\\
4.Proponer una interfaz para la captura y visualización (consulta)  de datos.\\
5.Proponer una estructura XML para almacenar los datos.\\
6.Implementar  la  aplicación  e  incorporarla  al  sistema  web  existente:  gestión  de  una  memoria corporativa.
\# \# \# \# \# \# \# \# \# \# \# \# \# \# \# \# \# \# \# \# \# \# \# \# \# \# \# \# \# \# \# \# \# \# \# \# \# \# \# \# \# \# \# \# \# \# \# \# \# \# \# \# \# \# \# \# \# \# \# \# \# \# \# \#\\\\
Como  primer  paso  se  realizó  una  investigación bibliográfica con  respecto a  la  familia  de problemas de coloración de gráficas, con lo cual se determinó que los problemas de coloración de  gráficas  mínima,  coloración  de  gráficas  débiles,  coloración  equitativa  y  en  específico  el problema de coloración robusta son las más utilizadas de esta familia.\\ 
A su vez se investigó los algoritmos utilizados para darles solución y sus aplicaciones.Los algoritmos más utilizados fueron: GRASP, búsqueda Tabú, algoritmos genéticos, búsqueda de vecindarios variable, recocido simulado, algoritmos bio-inspirados y búsqueda dispersa.\\ 
Por lo  cual  ser  realizo  una  investigación  sobre  su  funcionamiento,  características  principales, desempeño y si se han comparado con el algoritmo de búsqueda dispersa.\\ 
Se prosiguió a investigar de lleno el algoritmo de búsqueda dispersa, en la cual se enfatizó en las  recomendaciones  para  su  correcta  implementación,  como  también  las  distintas  estrategias para utilizar en los métodos del algoritmo y de esta forma generar una propuesta que se adapte a las necesidades del modelo de coloración de gráficas suaves y a su vez sea confiable, eficaz y eficiente.\\ 
Para garantizar estas características se investigaron instancias que se pudieran utilizar para  probar  el  algoritmo  propuesto,  pero  primero  se  concluyó  que  se  propondrían  instancias propias  para  de  esta  menara  tener  un  control  y  garantizar  que  se  implementaran  las  mejores estrategias para cada método del algoritmo, al obtener el algoritmo con las mejores estrategias se comparó y probo con instancias para el problema de coloración robustalas cuales ya habían sido resultas con anterioridad con el algoritmo de búsqueda dispersa y otros algoritmo que se investigaron,  con  el  fin  de  garantizar    la  confiabilidad,eficiencia  y  eficacia  del  algoritmo propuesto
