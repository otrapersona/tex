\section {Metodología científica}

ejemplo tesis master uam:

http://tesiuami.izt.uam.mx/uam/aspuam/presentatesis.php?recno=19133\&docs=UAMI19133.pdf

1. Revision y clasificacion de la literatura sobre trabajos relacionados.

2. Elaboracion de hipotesis y diseno de una estrategia de distribucion de contenido.

3. Desarrollo de un protocolo con base en la estrategia disenada.

4. Validacion del protocolo mediante simulacion.

5. Elaboracion de conclusiones.

6. Comunicacion idonea de resultados.

\# \# \# \# \# \# \# \# \# \# \# \# \# \# \# \# \# \# \# \# \# \# \# \# \# \# \# \# \# \# \# \# \# \# \# \# \# \# \# \# \# \# \# \# \# \# \# \# \# \# \# \# \# \# \# \# \# \# \# \# \# \# \# \# \# \# \# \# \# \# \# \# \# \# \# \# \# \# \# \# \# \# \# \# \# \# \# \# \# \# \# \# \# \# \# \# \# \# \# \# \# \# \# \# \# \# \# \# \# \# 

ejemplo 2

1.3.Metodología de investigaciónA lo largo de esta sección, mostraremos el proceso de investigación que hemos seguido para desarrollar este estudio, el cual se compone de las siguientes etapas (véase figura 1.3)

Formulación del proyecto de investigación. Durante esta etapa se define la problemática aresolver y se plantean los objetivos a cumplir, además de establecer una planeación de lasactividades a realizar a lo largo del estudio con base en una correcta delimitación del tema ainvestigar. Adicionalmente, se construye una hipótesis, la cual será sometida a un proceso deaceptación o refutación

Revisión sistemática de la literatura. A través de esta actividad, se estudian las diferentestécnicas que utilizan un modelo de mitigación de falsos positivos utilizando el enfoque deaprendizaje maquinal. También, se identifican y se analizan las principales etapas que componena cada técnica (y que mejor se adapten al desarrollo de nuestra propuesta), las herramientas desoftware utilizadas, los casos de estudio explorados y las medidas de desempeño encargadas deevaluar dichas técnicas

Diseño de la propuesta. Con base en el conocimiento adquirido en la etapa anterior, se diseñauna propuesta cuyo objetivo es aumentar el número de defectos relevantes descubiertos antes dellegar a la fase de pruebas dentro del PDS. Dicha propuesta se basa en las diferentes etapas quelos autores utilizan en la creación de sus modelos y en las diversas propiedades de sus AAIT1,además de incorporar nuevas características. La metodología training and testing [22] es parteesencial en el desarrollo de esta etapa.

Evaluación de la propuesta. Una vez que la propuesta ha sido diseñada, se vuelve necesarioimplementarla con el fin de realizar una evaluación de la misma y obtener una serie de resultadosque nos brinden información acerca del desempeño de todos los modelos generados y en general,del comportamiento de nuestra AAIT. Durante esta etapa, se genera, se identifica, se analiza y seselecciona la información necesaria para la construcción y evaluación de modelos de clasificaciónde alertas basados en diferentes algoritmos de aprendizaje maquinal. Todas estas actividades sonla parte experimental de la propuesta.

Análisis de resultados. Con base en la selección de los mejores modelos (es decir, aquellos quecuentan con el mayor poder predictivo), se realiza un análisis comparativo del desempeñologrado por el conjunto de modelos propios o internos al proyecto y el conjunto de modelosexternos al mismo. Posteriormente, se estudia la fiabilidad de incorporar una técnica como lanuestra dentro del PDS; las ventajas, desventajas e implicaciones de la propuesta son discutidas alo largo de esta etapa.