\usepackage[utf8]{inputenc}
\usepackage[spanish]{babel}
\author{Las mil secciones son para no perderme durante la escritura}
\title{}
%\date{\today}
\date{}
%>>> Mate, símbolos
\usepackage{amsmath}
\usepackage{amssymb}
\usepackage{amsfonts}
\usepackage{mathtools}
\usepackage{newtxmath} % fuente?
\usepackage{array}
\usepackage{multirow} % tablas
\usepackage{tabularx} % tablas
%<<< Mate, símbolos
%>>> Bibliografía
\usepackage{csquotes} % citar
%\usepackage[backend=biber,style=alphabetic,sorting=nyt]{biblatex} % OVERLEAF
%\addbibresource{bibfile.bib} % OVERLEAF
\usepackage[backend=bibtex,style=alphabetic]{biblatex} % genérico no-overleaf
\bibliography{pro} % genérico no-overleaf
%>>>  Bibliografía
%>>> Imágenes
\usepackage{graphicx} % insertar
\usepackage{float} % ubicación flotante
%>>> Imágenes
%>>> Clickable hypers
\usepackage{hyperref}
\hypersetup{
    hyperindex=true,
    linktocpage=false,
    colorlinks=true,
    allcolors=black,
    bookmarksnumbered=true,
    unicode=true,
}
%<<< Clickable hypers
%>>> notas verticales QUITAR AL FINAL
%\usepackage{marginnote}
\usepackage{geometry}
\geometry{marginparwidth=.45cm,marginparsep=.45cm} %NO TOCAR EL PRIMER .45, constante
\newcommand{\notad}[1]{\normalmarginpar\marginpar{\rotatebox{90}{$\uparrow$ #1}}}
%
\newcommand{\notai}[1]{\reversemarginpar\marginpar{\rotatebox{90}{$\downarrow$ #1}}}
%
\newcommand{\notadhi}[1]{\normalmarginpar\marginpar{\rotatebox{270}{$\downarrow$ #1}}}
%
\newcommand{\notaihi}[1]{\reversemarginpar\marginpar{\rotatebox{270}{$\uparrow$ #1}}}
%<<< notas verticales QUITAR AL FINAL
%CIRCULOS
\usepackage{tikz}
\usetikzlibrary{shapes}
%%\usepackage[export]{adjustbox}
\usepackage{subcaption}
\usepackage{wasysym}
\usepackage{multicol}
%
%\usepackage[table]{xcolor}