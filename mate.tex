Una \emph{variable aleatoria} es una función asignando un número real $\mathbb{R}$ a cada posible resultado de un experimento. Con una muestra en espacio $S$, una variable aleatoria $X$ asigna el valor numérico $X(s)$ a cada resultado posible $s$ del experimento. La aleatoriedad viene del hecho que tenemos un experimento aleatorio (con probabilidades descritas por la función de probabilidad $P$). Las variables aleatorias simplifican la notación y expanden la habilidad de cuantificar y resumir resultados de experimentos.

Se dice que una variable $X$ es discreta cuando si hay una lista finita de valores $a_,a_2,\ldots,a_n$ o un una lista infinita de valores $a_,a_2,\ldots$ de tal forma que $P(X=a_j$ para algún $j)=1$. Si $X$ es una variable aleatoria discreta, entonces el conjunto infinito o contable de valores $x$ tal que $P(X=x)$ se llama \emph{soporte} de $X$. En contraste una variable aleatoria continua puede tomar cualquier valor real en un intervalo.