\subsection{Mate dos}
La covarianza mide qué tanto o tan poco las dos variables aleatorias cuyos valores esperados existen y son positivos tienen dependencia lineal, denotada $cov(X,Y)$ la covarianza de $X$ y $Y$ es definida como

\begin{equation}
cov(X,Y)=E[(X-EX)(Y-EY)]

\end{equation}
Cuando no se tiene una referencia para usar la covarianza, tiene sentido escalarla de acuerdo a la desviación estándar de las variables\cite{mat17}

\begin{equation}
p(X,Y)=\frac{cov(X,Y)}{\sqrt{var(X)}\sqrt{var(Y)}}

\end{equation}
denotada $corr(X,Y)$ de esta es la \emph{correlación} de $X$ y $Y$. Un coeficiente de relación $p = 0$ indica que no hay relación.
Representamos la dependencia entre dos variables, en el que una aumenta o disminuye cuando la otra cambia con el \emph{coeficiente de correlación} $p$
\begin{equation}
p=\frac{Cov(Y_1,Y_2)}{\sigma_1\sigma_2}
\end{equation}