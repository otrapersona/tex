\subsection{Mate dos}
Representamos la dependencia entre dos variables, en el que una aumenta o disminuye cuando la otra cambia con la \emph{covarianza}
\begin{equation}
cov(X,Y)=E[(X-EX)(Y-EY)],
\end{equation}
cuando no se tiene referencia para la covarianza es conveniente escalarla de acuerdo a su desviación estándar, esto recibe el nombre de  \emph{coeficiente de correlación}
\begin{equation}
p=\frac{Cov(X,Y)}{\sigma_1\sigma_2}.
\end{equation}
Cuando tenemos un modelo que relaciona $E(Y)$ como una función lineal de de $\beta_0$ y $\beta_1$ únicamente, se tiene un modelo de regresión lineal \emph{simple}
\begin{equation}
E(Y)=\beta_0+\beta_1x\label{EQ:MRLS}
\end{equation},
cuando más de una variable independiente de interés, por ejemplo $x_1,x_2,\ldots,x_n$, se utiliza una generalización del modelo (\ref{EQ:MRLS}) \emph{múltiple}
\begin{equation}
E(Y)=\beta_0+\beta_1x+\ldots+\beta_nx_n
\end{equation}