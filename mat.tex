\subsection{Mate dos}
Representamos la dependencia entre dos variables, en el que una aumenta o disminuye cuando la otra cambia con la \emph{covarianza}
\begin{equation}
cov(X,Y)=E[(X-EX)(Y-EY)],
\end{equation}
cuando no se tiene referencia para la covarianza es conveniente escalarla de acuerdo a su desviación estándar, esto recibe el nombre de  \emph{coeficiente de correlación}
\begin{equation}
p=\frac{Cov(X,Y)}{\sigma_1\sigma_2}.
\end{equation}
Cuando tenemos un modelo que relaciona $E(Y)$ como una función lineal de de $\beta_0$ y $\beta_1$ únicamente, se dice que es un modelo de regresión lineal \emph{simple}
\begin{equation}
E(Y)=\beta_0+\beta_1x,
\end{equation}
cuando se tiene más de una variable independiente es un modelo de regresión lineal \emph{ múltiple}
\begin{equation}
E(Y)=\beta_0+\beta_1x+\ldots+\beta_kx_k
\end{equation}